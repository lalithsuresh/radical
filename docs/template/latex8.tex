%% LyX 1.6.5 created this file.  For more info, see http://www.lyx.org/.
%% Do not edit unless you really know what you are doing.
\documentclass[10pt,twocolumn,english,times]{article}
\usepackage[T1]{fontenc}
\usepackage[latin9]{inputenc}
\pagestyle{empty}

\makeatletter
%%%%%%%%%%%%%%%%%%%%%%%%%%%%%% Textclass specific LaTeX commands.
\newenvironment{lyxcode}
{\par\begin{list}{}{
\setlength{\rightmargin}{\leftmargin}
\setlength{\listparindent}{0pt}% needed for AMS classes
\raggedright
\setlength{\itemsep}{0pt}
\setlength{\parsep}{0pt}
\normalfont\ttfamily}%
 \item[]}
{\end{list}}

%%%%%%%%%%%%%%%%%%%%%%%%%%%%%% User specified LaTeX commands.

%  $Description: Author guidelines and sample document in LaTeX 2.09$ 
%  $Author: ienne $
%  $Date: 1995/09/15 15:20:59 $
%  $Revision: 1.4 $
\usepackage{latex8}\usepackage{times}

%\documentstyle[times,art10,twocolumn,latex8]{article}
%------------------------------------------------------------------------- 
% take the % away on next line to produce the final camera-ready version 


%------------------------------------------------------------------------- 

\makeatother

\usepackage{babel}

\begin{document}

\title{Really Awesome Distributed Internet Calendar (RADICAL)}


\author{Lalith Suresh P.\\
DEI\\
Instituto Superior Tecnico\\
Lisbon, Portugal\\
suresh.lalith@gmail.com\\
\and Marcus Ljungblad\\
DEI\\
Insituto Superior Tecnico\\
Lisbon, Portugal\\
marcus@ljungblad.nu\and Bruno Pereira\\
DEI\\
Insituto Superior Tecnico\\
Lisbon, Portugal\\
brunopereir4@gmail.com}

\maketitle
\thispagestyle{empty}
\begin{abstract}
Shared calendar systems like Google Calendar are known to be an effective
way for people to schedule and coordinate events. In this project,
we design and implement PADICal, a peer-to-peer based distributed
calendar system. PADICal allows clients to make event reservations
amongst themselves in an almost decentralised manner with minimal
assistance from a central server.
\end{abstract}

\section{Introduction}

The aim of this project is to design, implement and evaluate a shared
calendar system which has the following components:
\begin{itemize}
\item Multiple clients, each with their own calendars, who may contact one
another to schedule events together.
\item A centralised server which holds usernames and provides clients a
sequence number service.
\end{itemize}
Section 


\section{System Architecture}

Since there is a Client and Server entity for this system, we describe
the architecture of each separetely. One of the main design goals
is to allow easy testability and debugging facilities within the system
in order to ease development. To a certain degree, we hope to achieve
this through classic 'printf' style debugging. Every class in PADICal
inherits from a class \texttt{PadiCalObject}, which has two virtual
methods named \texttt{Debug() }and \texttt{UnitTests()}. The former
performs pretty printing of internal state of an object, and follows
the flow of logic through the stack whereas the latter is written
during the development phase to ensure methods do not misbehave when
changes are introduced later. All sub classes have to implement these
methods depending on the kind of methods and state information they
may hold. The \texttt{Debug()} method of different objects comprising
of a client or server can be enabled via a configuration file or through
some interfaces we can provide to the UI of the client.

The client and the server architecture has been logically decomposed
into 3 layers each. From top to down, they are as follows: the interface
layer, the services layer, and the communications layer. In the sections
below, we first describe the client, the server, and the components
of each of the 3 layers that they comprise of.


\subsection{Client}


\subsection{Server}
\begin{lyxcode}

\end{lyxcode}
\bibliographystyle{latex8}
\bibliography{latex8}

\end{document}
